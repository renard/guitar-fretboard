\documentclass[a4paper]{article}

\usepackage[english]{babel}
\usepackage{fontenc}
\usepackage[utf8]{inputenc}

\usepackage{tcolorbox}
\tcbuselibrary{listings,documentation}
\tcbuselibrary{documentation,listings,minted}
\tcbuselibrary{breakable}
\tcbuselibrary{skins,raster}
\tcbset{%
  listing engine=minted,%
  index actual=~,
}
\usepackage{url}
\usepackage{minted}
\usepackage{calc}
\usepackage{makeidx}

\usepackage{multicol}
\tcbset{
  doc-code/.style = {%
    boxrule=0.1pt,
    colframe=red!75!black,
    left=8mm,
    width=\linewidth,
    breakable=true,
    colback=white,
    enhanced,
  },
}


\newminted[macrocode]{latex}{
  breaklines,
  breakafter=()/+-\,
  baselinestretch=0.5,
  fontsize=\footnotesize,
  linenos,
  numbersep=3mm,
  firstnumber=last}
\BeforeBeginEnvironment{macrocode}{%
  \begin{tcolorbox}[doc-code]
  }%
\AfterEndEnvironment{macrocode}{
\end{tcolorbox}
}%


\newminted[example]{latex}{
  breaklines,
  breakafter=()/+-\,
  baselinestretch=0.5,
  fontsize=\footnotesize}
\BeforeBeginEnvironment{example}{%
  \begin{tcolorbox}[doc-code]}
\AfterEndEnvironment{example}{
\end{tcolorbox}
}%

\usepackage{guitar-fretboard}

\makeindex

\usepackage[super]{nth}

\newcommand{\pkg}[1]{\texttt{#1}}

\newif\ifintroduction
\introductiontrue

\begin{document}

\title{the \pkg{guitar-fretboard} package }
\author {Sébastien Gross <seb -at- chezwam -dot- org>}
\date{This file describes version \fbfileversion\  (\fbfiledate)}
\maketitle
\tableofcontents

\ifintroduction

\section{Introduction}

The \pkg{guitar-fretboard} package can be used to generate some nice-looking
guitar fretboard diagrams. Those diagrams can highlight scales, arpeggios or
more generic intervals on the guitar neck.

It comes with all battery included to allow you to:

\begin{itemize}
  \item Create scale diagrams in any tonality for any strung instruments
    (including guitar, bass, ukulele) in both right and left handed
    configuration.
  \item Transpose notes or scales.
  \item Create diagrams in alternate tuning.
  \item Create generic diagrams with normal, shaded and highlighted notes.
  \item and more.
\end{itemize}

As an example here is the Major scale in key of \pG:

\begin{tcblisting}{doc-code,text above listing}
  \begin{fb}[frets before = 2, frets after = 2,
      legend text = {Major scale (1,2,3,4,5,6,7)},
      fret numbers visible]
    \foreach \i in {1, 2, 3, 4, 5, 6, 7} {
      \note[lower]{\i}
    }
  \end{fb}
\end{tcblisting}

The major scale consists of following intervals from the tonic (root) note:
perfect unison, major second, major third, perfect fourth, perfect fifth,
major sixth and major seventh. These intervals are represented as
\brackets{1,2,3,4,5,6,7}. Each interval has its own color for an easier
lookup and memorization.

The note names can be displayed instead of intervals. The \pF Major scale is
composed of \pF,\pG,\pA,\pBb,\pC,\pD and \pE:

\begin{tcblisting}{doc-code,text above listing}
  \begin{fb}[frets before = 2, frets after = 2,
      transpose = 5,
      transpose pitch,
      legend text = {F Major scale (\pF, \pG, \pA, \pBb, \pC, \pD, \pE)},
      fret numbers visible]
    \foreach \i in { C, D, E, F, G, F, A, B} {
      \note[lower]{\i}
    }
  \end{fb}
\end{tcblisting}

The \pF scale is the same as the \pC one but we raise it by 5 semi
tones. \pkg{guitar-fretboard} is able to convert the pitch name as well. The
pitch transposition is based on semi-tones and may not be always accurate in
terms of harmony since not intervals are used in the process.

Chords can also be displayed. In that case you can easily see new \pC chord
shapes showing up. If you are playing in \pC Major, the scale can also be
displayed. This helps you to find new embellishments when soloing:

\begin{tcblisting}{doc-code,text above listing}
  \begin{fb}[frets before = 2, frets after = 2,
      legend text = {C Chord (and C major scale)},
      fret numbers visible]
    \foreach \i in {C, E, G} {
      \note{\i}
    }
    \foreach \i in {D, F, A, B} {
      \note[shade]{\i}
    }
  \end{fb}
\end{tcblisting}

The classical \pC open chord can also be displayed. In that case the
\meta{limit} option is used to only place the G on the \nth{3} string.

\begin{tcblisting}{doc-code,text above listing}
  \begin{fb}[frets min = 0, frets max = 3,
      legend text = {\pC Chord},
      fret numbers visible]
    \note{C} \note{E} \note[limit={3}]{G}
  \end{fb}
\end{tcblisting}

The same chord in open \pG{} tuning. Again \pkg{guitar-fretboard} does all
the magic for you as long as you ask nicely.

\begin{tcblisting}{doc-code,text above listing}
  \begin{fb}[frets min = 0, frets max = 3,
      tuning={2,7,2,7,11,2},
      legend text = {\pC Chord in open \pG tuning},
      fret numbers visible]
    \note{C} \note[limit={1,4}]{E} \note{G}
  \end{fb}
\end{tcblisting}

And the \pC M7 arpeggio within the whole scale in open \pG tuning:

\begin{tcblisting}{doc-code,text above listing}
  \begin{fb}[frets before = 2, frets after = 2,
      tuning={2,7,2,7,11,2},
      legend text = {\pC M7 arpeggio in open \pG tuning},
      fret numbers visible]
    \foreach \i in {C, E, G, B} {
      \note{\i}
    }
    \foreach \i in {D, F, A} {
      \note[shade]{\i}
    }
  \end{fb}
\end{tcblisting}



If you want to display a scale mode you can also emphasis the characteristic
notes (the note that colors the mode, you want to play that note). Here is
an example with a generic Phrygian mode:

\begin{tcblisting}{doc-code,text above listing}
  \begin{fb}[frets min=3, frets max=6,
      transpose = -5,
      legend text = {Phrygian scale}]
    \foreach \i in {1,b2,b3,5,b6,b7} {
      \note{\i}
    }
    \foreach \i in {b2,b3} {
      \note[limit={1},shade]{\i}
    }
    \note[highlight]{4}
  \end{fb}
\end{tcblisting}
\fi

\section{Definitions}

Under the hood all computation is done using semitones (from 0 to 11)
relative to \pC.


\begin{center}
  \begin{tabular}[t]{clll}
    Semitone & Note & Internal name & Note command\\
    \hline
    0  & \pC           & C        & \cs{pC}             \\
    1  & \pCS{} / \pDb & CS / Db  & \cs{pCS} / \cs{pDb} \\
    2  & \pD           & D        & \cs{pD}             \\
    3  & \pDS{} / \pEb & DS / Eb  & \cs{pDS} / \cs{pEb} \\
    4  & \pE           & E        & \cs{pE}             \\
    5  & \pF           & F        & \cs{pF}             \\
    6  & \pFS{} / \pGb & FS / Gb  & \cs{pFS} / \cs{pGb} \\
    7  & \pG           & G        & \cs{pG}             \\
    8  & \pGS{} / \pAb & GS / Ab  & \cs{pGS} / \cs{pAb} \\
    9  & \pA           & A        & \cs{pA}             \\
    10 & \pAS{} / \pBb & AS / Bb  & \cs{pAS} / \cs{pBb} \\
    11 & \pB           & B        & \cs{pB}             \\
  \end{tabular}
\end{center}

From here we won't distinguish a note from an interval since they are
defined from semitones.

\subsection{Fretboard definition}

\begin{docEnvironment}{fb}{\oarg{options}}

This is the main environment to create a fretboard. Several \oarg{options}
can be passed to customize the environment.

\begin{docKey}[fb][]{fretboard}{=\meta{style}}{}
  Style applied to the main \pkg{tikz} environment.
\end{docKey}

\begin{docKey}[fb][]{scale}{=\meta{scale}}{0.3}
  Scale factor applied to the \pkg{tikz} environment.
\end{docKey}

\begin{docKey}[fb][]{frets min}{=\meta{frets min}}{0}
  Lowest fret to be displayed.
\end{docKey}

\begin{docKey}[fb][]{frets max}{=\meta{frets max}}{12}
  Highest fret to be displayed.
\end{docKey}

\begin{docKey}[fb][]{frets before}{=\meta{frets before}}{0}
  Number of frets displayed before the lowest one. This area will be
  shaded. This is useful for a larger view of the neck while focusing on
  small part.
\end{docKey}

\begin{docKey}[fb][]{frets after}{=\meta{frets after}}{0}
  Same as \refKey{/fb/frets before} but after highest fret.
\end{docKey}

\begin{docKey}[fb][]{frets offset}{=\meta{frets offset}}{0.5}
  Offset to shift the frets on the x axis. You probably don't want to play
  with this setting.
\end{docKey}

\begin{docKey}[fb][]{tuning}{=\meta{tuning}}{{4, 9, 2, 7, 11, 4}}
  The instrument tuning from lowest (\nth{6}) to highest (\nth{1})
  string. Keep in mind that all is matter of semitones relative to \pC. For
  a bass you probably want to use \brackets{4, 9, 2, 7}, for Dropped \pD, 
  \brackets{2, 9, 2, 7, 11, 4}, etc.
\end{docKey}

\begin{docKey}[fb][]{transpose}{=\meta{semitones}}{0}
  Number of semitones to apply to current fretboard definition.
\end{docKey}

\begin{docKey}[fb][]{transpose pitch}{=\meta{transpose pitch}}{false}
  If true the note pitches will also be transposed. Note this is non-sense
  for intervals.
\end{docKey}

\begin{docKey}[fb][]{string}{=\meta{style}}{0}
  String drawing style.
\end{docKey}

\begin{docKey}[fb][]{string width}{=\meta{string width}}{0.5pt}
  Initial width of the highest (thinnest) string (\nth{1}).
\end{docKey}

\begin{docKey}[fb][]{string factor}{=\meta{string factor}}{0.5pt}
  A growth factor to be applied when drawing strings from the highest
  (\nth{1}) to the lowest (biggest one).
\end{docKey}

\begin{docKey}[fb][]{frets}{=\meta{style}}{}
  Style for drawn frets.
\end{docKey}

\begin{docKey}[fb][]{fret numbers}{=\meta{style}}{}
  Style for fret numbers.
\end{docKey}

\begin{docKey}[fb][]{fret numbers position}{=\meta{fret numbers position}}{0.75}
  Fret numbers position below fretboard.
\end{docKey}

\begin{docKey}[fb][]{note}{=\meta{style}}{}
  Default style for note (See \refKey{/notes/NOTE/style}).
\end{docKey}

\begin{docKey}[fb][]{split note}{=\meta{style}}{}
  Default style for split note (See \refKey{/notes/NOTE/split style}).
\end{docKey}

\begin{docKey}[fb][]{highlight note}{=\meta{style}}{}
  Default style for highlighted note (See \refKey{/notes/NOTE/highlight style}).
\end{docKey}

\begin{docKey}[fb][]{overlay}{=\meta{style}}{}
  Style for overlay mask applied over \refKey{/fb/frets before} and
  \refKey{/fb/frets after}.
\end{docKey}

\begin{docKey}[fb][]{legend text}{=\meta{text}}{}
  The fretboard legend.
\end{docKey}

\begin{docKey}[fb][]{legend}{=\meta{style}}{}
  The fretboard legend style.
\end{docKey}


\end{docEnvironment}


\begin{docCommand}{note}{\oarg{options}\marg{name}}
  Place note \meta{name} on the fretboard. This command only makes sense in
  a \refEnv{fb} environment.

  \oarg{options} are the same as \refCom{newnote} and can be overridden.

  Examples:
  \begin{example}
    % Display a squared unisson.
    \note[style/.append style={shape=rectangle}]{1}
  \end{example}
  
  \begin{example}
    % Display shaded notes
    \foreach \i in {D, F, A, B} {
      \note[shade]{\i}
    }
  \end{example}

  \begin{example}
    % Display a C open chord
    \note[style/.append style={rectangle}]{C}
    \note{E}
    \note[limit={3}]{G}
  \end{example}
  
\end{docCommand}


\subsection{Color definition}

(todo)

\subsection{Note definition}

\pkg{guitar-fretboard} comes with a lot of interval and pitch
definitions. Still you can add your owns.

\begin{docCommand}{newnote}{\oarg{options}\marg{name}}
  Creates a new note \meta{name} relatively from \pC. The new note can be
  referred by \meta{name} when displaying it on the fretboard. \meta{name}
  will be a part of a \pkg{PGF} key thus some characters such as sharp(\#)
  cannot be used. By an arbitrary convention a note is designed by
  \notename{X}, the flatten \notename{X}\accidental{b} and the sharped
  \notename{X}\accidental{S}.

  For a new note a new PFG tree is created under \docAuxKey[notes]{NOTE}.

  \begin{docKey}[notes/NOTE][]{semitones}{=\meta{semitones}}{0}
    Number of semitones from \pC.
  \end{docKey}

  \begin{docKey}[notes/NOTE][]{text}{=\meta{text}}{\marg{name}}
    Displayed note name. For pitches, you can use the \cs{pX} shortcuts such
    as \cs{pA}, \cs{pBb} \cs{pFS} and so on.
  \end{docKey}

  \begin{docKey}[notes/NOTE][]{lower text}{=\meta{lower text}}{}
    Same as \refKey{/notes/NOTE/text} but for lower part text.
  \end{docKey}

  \begin{docKey}[notes/NOTE][]{limit}{=\meta{limit}}{}
    Set a string limit when placing the notes. This can be useful when
    drawing a chord. You may only want to use this option when displaying
    note using \refCom{note}.
  \end{docKey}

  \begin{docKey}[notes/NOTE][]{style}{=\meta{style}}{}
    A style definition to be applied to the note when the lower part is
    omitted.
  \end{docKey}

  \begin{docKey}[notes/NOTE][]{split style}{=\meta{split style}}{}
    A style definition to be applied to the note when the lower part is
    displayed.
  \end{docKey}

  \begin{docKey}[notes/NOTE][]{highlight style}{=\meta{highlight style}}{}
    A style definition to be applied to emphasis the displayed note. This is
    useful for target or characteristic notes.
  \end{docKey}

  \begin{docKey}[notes/NOTE][]{lower}{=\meta{boolean}}{false}
    Set to true is you want to display both text and lower text.
  \end{docKey}

  \begin{docKey}[notes/NOTE][]{shade}{=\meta{boolean}}{false}
    Set to true if you want to shade the note to make it less important than
    others.
  \end{docKey}

  Examples:
  \begin{example}
    % Definition of C
    \newnote[semitones=0,
      text=\pC,
      style/.append style={fill=bg1, text=fg1}%
    ]{C}
  \end{example}

  \begin{example}
    % Definition of D#/Eb
    \newnote[semitones=3,
      text=\pDS,
      lower text=\pEb,,
      style/.append style={fill=bg2s, text=fg2s}%
    ]{DS}
  \end{example}  
\end{docCommand}


\begin{docCommand}{copynote}{\oarg{options}\marg{from}\marg{to}}
  Copies note from \meta{from} to \meta{to}.

  \oarg{options} are the same as \refCom{newnote}.

  Examples:
  \begin{example}
    % Definition of Unison (1) which is C.
    \copynote[text=1,]{C}{1}
  \end{example}
  
  \begin{example}
    % Definition of Major Third which is E. The inversed interval is a minor
    % sixth.
    \copynote[text=M3,lower text=m6]{E}{M3}
  \end{example}

\end{docCommand}




\end{document}
