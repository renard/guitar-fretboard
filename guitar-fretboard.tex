\documentclass[a4paper]{article}

\newif\ifintroduction
\introductiontrue
\newif\ifannexes
\annexestrue


\usepackage[english]{babel}
\usepackage{fontenc}
\usepackage[utf8]{inputenc}

\usepackage{tcolorbox}
\tcbuselibrary{listings,documentation}
\tcbuselibrary{documentation,listings,minted}
\tcbuselibrary{breakable}
\tcbuselibrary{skins,raster}
\tcbset{%
  listing engine=minted,%
  index actual=~,
}
\usepackage{url}
\usepackage{minted}
\usepackage{calc}
\usepackage{makeidx}

\usepackage{multicol}
\tcbset{
  doc-code/.style = {%
    boxrule=0.1pt,
    colframe=red!75!black,
    %% left=8mm,
    halign=flush center,
    width=\linewidth,
    breakable=true,
    colback=white,
    enhanced,
  },
  warning/.style =  {%
    colback=red!5!white,
    colframe=red!45!white,
    boxrule=0.5pt,
  },
}


\newminted[macrocode]{latex}{
  breaklines,
  breakafter=()/+-\,
  baselinestretch=0.5,
  fontsize=\footnotesize,
  linenos,
  numbersep=3mm,
  firstnumber=last}
\BeforeBeginEnvironment{macrocode}{%
  \begin{tcolorbox}[doc-code]
  }%
\AfterEndEnvironment{macrocode}{
\end{tcolorbox}
}%


\newminted[example]{latex}{
  breaklines,
  breakafter=()/+-\,
  baselinestretch=0.5,
  fontsize=\footnotesize}
\BeforeBeginEnvironment{example}{%
  \begin{tcolorbox}[doc-code]}
\AfterEndEnvironment{example}{
\end{tcolorbox}
}%

\usepackage{guitar-fretboard}

\makeindex

\usepackage[super]{nth}

\newcommand{\pkg}[1]{\texttt{#1}}

\usepackage{longtable}


\newcommand{\mypitchline}[1]{
  \texttt{#1} &
  \FBgetNote{#1}{degree} &
  \FBgetNote{#1}{semitones} &
  \FBgetNote{#1}{tex} &
  \FBgetNote{#1}{tex small} &
  \texttt{\FBgetNote{#1}{reversed}} &
  \texttt{\textbackslash{P#1}}
  \\
}


\newcommand{\myintervalline}[1]{
  \texttt{#1} &
  %% \texttt{\FBgetNote{#1}{aliases}} &
  \FBgetNote{#1}{degree} &
  \FBgetNote{#1}{semitones} &
  \FBgetNote{#1}{name} &
  %% \csname I#1\endcsname &
  \FBgetNote{#1}{tex} &
  \FBgetNote{#1}{tex small} &
  \FBgetNote{#1}{reversed} 
  \\
}


\newcommand{\mychordline}[1]{
  \texttt{#1} &
  \getScale{#1}{name} &
  \showChord{C}{#1} &
  \showChordFs{C}{#1} &
  \getScaleIntervals{#1}
  \\
}

\makeatletter
\newcommand{\ListSemitones}[1]{%
  \begin{itemize*}[%
      afterlabel={},
      before={(},
      after={)},
      label={},
      mode=unboxed,
      itemjoin={,\space}]%
    \edef\@current@tuning{\pgfkeysvalueof{/fb/tunings/#1}}%
    \foreach \i in \@current@tuning{%
      \item\i
    }%
  \end{itemize*}%
}%

\newcommand{\FBnotesFromList}[1]{%
    \edef\@current@tuning{\pgfkeysvalueof{/fb/tunings/#1}}%
    \foreach \i in \@current@tuning{%
      \@transposepitch{C}{\i}%
      \csname p\newpitch\endcsname%
    }%
}%
\makeatother

\newcommand{\mytuningline}[1]{
  \texttt{#1} &       
  \FBnotesFromList{#1} &
  \ListSemitones{#1}
  \\
}

\usepackage{colortbl}
\newcommand{\mycolorline}[1]{
  \texttt{#1} &
  {\cellcolor{RGBbg#1}{\textcolor{fg#1}{\texttt{RGBbg#1} / \texttt{fg#1}}}} &
  {\cellcolor{bg#1}{\textcolor{fg#1}{\texttt{bg#1} / \texttt{fg#1}}}}
  \\
}



\begin{document}

\title{the \pkg{guitar-fretboard} package }
\author {Sébastien Gross <seb -at- chezwam -dot- org>}
\date{This file describes version \fbfileversion\  (\fbfiledate)}
\maketitle
\tableofcontents


\begin{tcolorbox}[warning,before={\vspace{1cm}}]
  \pkg{guitar-fretboard} is still in an early development stage. High-level
  functions are stable enough to be used. Low-level functions and storage
  are subject to change, especially the note definition. That said with
  high-level functions you can yet do a lot of things.
\end{tcolorbox}


\ifintroduction

\section{Introduction}

The \pkg{guitar-fretboard} package can be used to generate some nice-looking
guitar fretboard diagrams. Those diagrams can highlight scales, arpeggios or
more generic intervals on the guitar neck.

It comes with batteries included to allow you to:

\begin{itemize}
  \item Create scale diagrams in any tonality for any strung instruments
    (including guitar, bass, ukulele) in both right and left handed
    configuration.
  \item Transpose notes or scales.
  \item Create diagrams in alternate tuning.
  \item Create generic diagrams with normal, shaded and highlighted notes.
  \item and more.
\end{itemize}

As an example here is the Major scale in key of \pC:

\begin{tcblisting}{doc-code,text above listing}
  \begin{fretboard}[frets before = 2, frets after = 2,
      title = {Major scale (1, 2, 3, 4, 5, 6, 7)},
      scale=0.35, fret numbers]
    \foreach \i in {1, 2, 3, 4, 5, 6, 7} {
      \FBnote[split]{\i}
    }
  \end{fretboard}
\end{tcblisting}

The major scale consists of following intervals from the tonic (root) note:
perfect unison, major second, major third, perfect fourth, perfect fifth,
major sixth and major seventh. These intervals are represented as
\brackets{1, 2, 3, 4, 5, 6, 7}. Each interval has its own color for an
easier lookup and memorization.

The note names can be displayed instead of intervals. The \pF Major scale is
composed of \pF, \pG, \pA, \pBb, \pC, \pD and \pE. The \pF scale is the same
as \pC but raised up by 5 semitones. \pkg{guitar-fretboard} is able to
transpose and convert the pitch name accordingly. Please note that the pitch
transposition is semitone-based and intervals are not used in the
process. Hence note names may not be always accurate in terms of harmony.

\begin{tcblisting}{doc-code,text above listing}
  \begin{fretboard}[frets before = 2, frets after = 2,
      transpose = 5,
      transpose pitch,
      title = {\pF Major scale\\\Large (\pF, \pG, \pA, \pBb, \pC, \pD, \pE)},
      scale=0.35,
      fret numbers]
    \foreach \i in { C, D, E, F, G, F, A, B} {
      \FBnote[split]{\i}
    }
  \end{fretboard}
\end{tcblisting}



Chords can also be displayed. In that case you can easily see new \pC chord
shapes showing up. If you are playing in \pC Major, the scale can also be
displayed. This helps you to find new embellishments when soloing:

\begin{tcblisting}{doc-code,text above listing, center upper}
  \begin{fretboard}[frets before = 2, frets after = 2,
      title = {\pC Chord (and \pC major scale)},
      scale = 0.35,
      fret numbers]
    \foreach \i in {C, E, G} {
      \FBnote{\i}
    }
    \foreach \i in {D, F, A, B} {
      \FBnote[shade]{\i}
    }
  \end{fretboard}
\end{tcblisting}

The \pC chord can also be displayed in a more classical way. Use the
\meta{chord} option for that in addition with the \meta{limit} option which
helps to only show the \pG on the \nth{3} string.

\begin{tcblisting}{doc-code,text above listing}
  \begin{fretboard}[frets min = 0, frets max = 3, fret numbers,
      title = {\pC}, chord, scale=0.35]
    \FBnote{C} \FBnote{E} \FBnote[limit={3}]{G}
  \end{fretboard}
\end{tcblisting}

How this \pC chord looks like when the guitar is open \pG tuned? Again
\pkg{guitar-fretboard} does all the magic for you as long as you ask nicely.

\begin{tcblisting}{doc-code,text above listing}
  \begin{fretboard}[frets min = 0, frets max = 3,
      chord, tuning = \FBtuning{guitar/open g},
      title = {\pC Chord (open \pG)},
      fret numbers, scale=0.35]
    \FBnote{C} \FBnote[limit={1,4}]{E} \FBnote{G}
  \end{fretboard}
\end{tcblisting}

As you can see, setting the \meta{tuning} option updates the whole
fretboard. This is very easy to create some new diagrams in different
tunings.

Now let's figure out how a left-handed guitarist can practice the \pC\unskip{}M7
arpeggio (and the whole major scale) in open \pG tuning:

\begin{tcblisting}{doc-code,text above listing}
  \begin{fretboard}[frets before = 2, frets after = 2,
      tuning = \FBtuning{guitar/open g}, left handed,
      title = {\pC\unskip{}M7 arpeggio in open \pG tuning},
      fret numbers, scale=0.35]
    \foreach \i in {C, E, G, B} {
      \FBnote{\i}
    }
    \foreach \i in {D, F, A} {
      \FBnote[shade]{\i}
    }
  \end{fretboard}
\end{tcblisting}


If you want to display a scale mode you can also emphasis the characteristic
note (the note that colors the mode, you want to play this particular
note). Here is an example with a generic Phrygian mode:

\begin{tcblisting}{doc-code,text above listing}
  \begin{fretboard}[frets min=2, frets max=6,
      transpose = -5,
      title = {Phrygian scale}, scale=0.35]
    \foreach \i in {1,b2, 3, 4, 5,b6,b7} {
      \FBnote{\i}
    }
    \FBnote[highlight]{b2}
    \FBnote[highlight, shade, limit={1}]{b2}
  \end{fretboard}
\end{tcblisting}

\fi


\section{Definitions}

Under the hood all computations are made using semitones (from 0 to 11)
relative to \pC. In the annexes you can find all pitches (notes), intervals
and colors defined in \pkg{guitar-fretboard}.

\subsection{Styles}

Several global \pkg{tikz} styles are defined and can be overridden if they
don't fit your needs. All their default values are not listed here, please
have a look to the \pkg{guitar-fretboard} definition file if you are curious
about it.

The main point is that you can override them globally using the \cs{tikzset}
command or just when you need in the \refEnv{fretboard} environment or when using
the \refCom{FBnote} command.

\begin{docKey}[tikz][]{fretboard}{=\meta{style}}{}
  The style to be passed to the Tikz picture. From here you can change the
  background color and many other things supported by the \pkg{tikzpicture}
  environment.

  Both \meta{xscale} and \meta{yscale} are set here for an optimal render in
  scale, chord and lefty modes. Don't change them unless you really known
  what you are doing. By default the \pkg{x} size is twice the \pkg{y} size
  in scale mode. In chord mode, the \pkg{x} \pkg{y} ratio is 1.2.
\end{docKey}

\begin{docKey}[tikz][]{fret}{=\meta{style}}{}
  This style is used to draw the frets. You can change it is you want
  colored frets or larger lines to draw the frets.
\end{docKey}

\begin{docKey}[tikz][]{fret numbers}{=\meta{style}}{}
  This style is used for the fret numbering. By default the font is given by
  the \cs{fretfont} command.
\end{docKey}

\begin{docKey}[tikz][]{fret numbers position}{=\meta{style}}{\texttt{\{ shift = \{(0, 0.75)\}\}}}
  This style is used is a tikz scope to shift the fret numbers. the default
  value should fit all cases. If you want to place the fret numbers
  somewhere else, you can use this option but you might need to try several
  values.

  Be careful with shift values since they are cumulative. Thus use
  $(0,~-0.75)$ to reset the position.
\end{docKey}

\begin{docKey}[tikz][]{string tuning}{=\meta{style}}{}
  This style is used for the string labeling. By default the font is given
  by the \cs{stringfont} command.
\end{docKey}

\begin{docKey}[tikz][]{string tuning position}{=\meta{style}}{\texttt{\{ shift = \{(-0.75, 1)\}\}}}
  Same as \refKey{/tikz/fret numbers position} but for the string labels.
\end{docKey}

\begin{docKey}[tikz][]{string}{=\meta{style}}{}
  Same as \refKey{/tikz/fret} but for the strings. You can control
  everything but the sting width which is overridden by both
  \refKey{/fb/string width} and \refKey{/fb/string factor}.
\end{docKey}

\begin{docKey}[tikz][]{overlay}{=\meta{style}}{}
  This style is used to fade out the \refKey{/fb/frets before} and
  \refKey{/fb/frets after} to help focus on the main part of the neck.
\end{docKey}

\begin{docKey}[tikz][]{title}{=\meta{style}}{}
  This style is used for the diagram title. By default the font is given
  by the \cs{labelfont} command.
\end{docKey}

\begin{docKey}[tikz][]{title position}{=\meta{style}}{\{ shift = \{(0, -0.75)\}\}}
  Same as \refKey{/tikz/fret numbers position} but for the diagram title.

  Be careful when you change it because in chord mode, $(-0.75,~0.75)$ is
  added to this value. Again you may need to experiment several values if
  you really want to customize the label position..
\end{docKey}

\begin{docKey}[tikz][]{note}{=\meta{style}}{}
  This style is used when drawing a note on the fretboard. By default the
  \cs{notefont} is used.
\end{docKey}

\begin{docKey}[tikz][]{note split}{=\meta{style}}{}
  This style is used when drawing a split note on the fretboard. Used in
  addition of \refKey{/tikz/note} to draw a circle split and reduce the font
  size.
\end{docKey}

\begin{docKey}[tikz][]{note shade}{=\meta{style}}{}
  This style is used when shading a non important note on the
  fretboard. Used in addition of \refKey{/tikz/note} to add an overlay to
  the note.
\end{docKey}

\begin{docKey}[tikz][]{note highlight}{=\meta{style}}{}
  This style is used when highlighting an important note on the
  fretboard. Used in addition of \refKey{/tikz/note} to mark to
  the note.
\end{docKey}


\subsection{Fretboard}

\begin{docEnvironment}{fretboard}{\oarg{options}}

  This is the main part of the \pkg{guitar-fretboard} engine which is
  responsible to draw the frets, the strings, the labels and put everything
  together.

  You can control the default settings by defining several options all
  relative to the \docAuxKey*{/fb} path.

  \begin{docKey}[fb][]{tuning}{=tuning}{\{\FBtuning{guitar/standard}\}}
    The instrument tuning. You can select a pre-defined tuning with the
    \refCom{FBtuning} command or provide your own. Each string is identified by
    a semitone value (relative to \pC). The strings order is from low to
    high.
  \end{docKey}

  \begin{docKey}[fb][]{frets min}{=0}{0}
    The value of the lowest fret on the main fretboard section.
  \end{docKey}

  \begin{docKey}[fb][]{frets max}{=12}{12}
    The value of the highest fret on the main fretboard section.
  \end{docKey}

  \begin{docKey}[fb][]{frets before}{=0}{0}
    Number of frets to extend the fretboard before the main part. This are
    will be shaded.
  \end{docKey}

  \begin{docKey}[fb][]{frets after}{=0}{0}
    Number of frets to extend the fretboard after the main part. This are
    will be shaded.
  \end{docKey}

  \begin{docKey}[fb][]{frets offset}{=0.5}{0.5}
    By default the frets are shifted from the grid to make note placement
    easier. Do not change this value or your fretboard will be mangled.
  \end{docKey}

  \begin{docKey}[fb][]{frets numbers}{=boolean}{false}
    Set to true to show the fret numbers.
  \end{docKey}

  \begin{docKey}[fb][]{string width}{=0.5pt}{0.5pt}
    The width of the higher string.
  \end{docKey}

  \begin{docKey}[fb][]{string factor}{=0.75pt}{0.75pt}
    The string width growing factor when drawing frets from highest to
    lower.
  \end{docKey}

  \begin{docKey}[fb][]{show tuning}{=boolean}{true}
    Set to false to hide the string labels at the top of the neck.
  \end{docKey}
  
  \begin{docKey}[fb][]{split}{=boolean}{false}
    Set to false to show 2 labels in the notes when applicable (only when
    \refKey{/fb/notes/NOTE/reversed} is set). If note is a pitch both flat
    and sharp pitch are shown such as \pEb and \pFS. In the case of an
    interval, the reversed interval is shown.
  \end{docKey}
  
  \begin{docKey}[fb][]{title}{=text}{}
    If not blank, the \docValue{text} content is used for the diagram title.
  \end{docKey}

  \begin{docKey}[fb][]{transpose}{=semitone}{0}
    Number of semitone to use to transpose the whole fretboard diagram. If
    your diagram is the \pC major scale and you want to show the \pF major
    scale, you will need to change all the notes you entered. Using the
    transposing option, \pkg{guitar-fretboard} will do the job for you by
    setting a 5-semitone transposition.
  \end{docKey}

  \begin{docKey}[fb][]{transpose pitch}{=boolean}{false}
    When transposing a pitched diagram you may want to display the correct
    note name on their frets. So you need to transpose pitches or let
    \pkg{guitar-fretboard} do the job for you. Be careful when you use
    pitches because the resulting names may be wrong (at least not correct
    in terms of harmony).
  \end{docKey}

  \begin{docKey}[fb][]{chord}{=boolean}{false}
    Display the diagram in chord mode (rotate the whole diagram by
    90\textsuperscript{o} counter-clockwise and change the fret ratio.
  \end{docKey}

  \begin{docKey}[fb][]{scale}{=ratio}{1}
    Scale ratio used to resize the diagram.
  \end{docKey}

  \begin{docKey}[fb][]{left handed}{=boolean}{false}
    Display the diagram for left-handed guitarists.
  \end{docKey}

\end{docEnvironment}

\subsection{Useful commands}

\begin{docCommand}{FBnote}{\oarg{options}\marg{id}}
  Place the note referenced by \marg{id} on the current fretboard. This
  command only makes sense in the \refEnv{fretboard} environment.

  You can override any parameter set in the \refCom{FBnewNote} command or
  any \docAuxKey{/tikz} style since \oarg{options} is processed by
  \cs{tikzset}.
\end{docCommand}

\begin{docCommand}{FBnoteAt}{\oarg{options}\marg{id}\marg{string}\marg{fret}}
  Similar to \refCom{FBnote} but place note \docValue{id} at \docValue{string}
  on \docValue{fret}.
\end{docCommand}


\begin{docCommand}{FBtuning}{\marg{id}}
  Retrieves the tuning referenced by \marg{id} under the
  \docAuxKey*{/fb/tunings} path. See annexes for all pre-defined tunings.
\end{docCommand}


\subsection{Other commands}

From here are some commands you don't need for you daily usage except if you
want to define your own notes.

\begin{tcolorbox}[warning]
Warning internal nomenclature is subject to change. Don't use command from
here unless you really know what you are doing.
\end{tcolorbox}


\begin{docCommand}{FBnewNote}{\oarg{options}\marg{id}}
  Create a new note identified by \marg{id}. It should respect the
  \pkg{pgfkeys} key syntax. It's better to only use alphanumerical
  characters. By an arbitrary convention, flats are replaced by \texttt{b}
  and sharps by \texttt{S}. Thus a \pCS is referenced as \texttt{CS} and
  \pDb by \texttt{Db}. See the annexes for all defined pitches and
  intervals.

  The newly created note will be placed under the
  \docAuxKey*{/fb/notes/NOTE} key path.
  
  In addition you can add some \oarg{options} to the new note. From here a
  pitch or an interval is referenced as a note.

  \begin{docKey}[fb/notes/NOTE][]{name}{=name}{}
    The full textual name of the note that can be used for specific usage.
  \end{docKey}

  \begin{docKey}[fb/notes/NOTE][]{degree}{=degree}{}
    The note degree in the \pC major scale reference (1 or \pC, 2 for \pD, 3
    for \pE, etc). the degree does not care about semitones, just about the
    note base name.
  \end{docKey}

  \begin{docKey}[fb/notes/NOTE][]{tex}{=tex}{}
    the note representation in a musical environment. This key is used by
    the \cs{pX} command where \texttt{X} is the note \marg{id}. See the
    annexes for further details.
  \end{docKey}

  \begin{docKey}[fb/notes/NOTE][]{tex small}{=tex}{}
    Same as \refKey{/fb/notes/NOTE/tex} but used for fretboard limited
    space. Flats and sharps are smaller (\FBgetNote{CS}{tex}{} and
    \FBgetNote{CS}{tex small}).
  \end{docKey}

  \begin{docKey}[fb/notes/NOTE][]{semitones}{=0}{0}
    Semitone distance from the \pC note. This is used to determine the
    note's place on the fretboard.
  \end{docKey}

  \begin{docKey}[fb/notes/NOTE][]{reversed}{=note ref}{}
    The reference to the reversed interval \marg{id} or equivalent note name.
  \end{docKey}

  From this point all other options are only useful when placing a note on
  the fretboard using the \refCom{FBnote} command to override default
  behaviors.

  \begin{docKey}[fb/notes/NOTE][]{limit}{={list, of, strings}}{}
    A list of strings (order does not matter) to limit the note placement on
    the fretboard. For example if you only want to place the note on both
    \nth{5} and \nth{2} string, use \brackets{2,5} (or \brackets{5,2}).
  \end{docKey}

  \begin{docKey}[fb/notes/NOTE][]{style}{={style}}{}
    A complement style definition to \refKey{/tikz/note}.
  \end{docKey}

  \begin{docKey}[fb/notes/NOTE][]{highlight style}{={style}}{}
    A complement style definition to \refKey{/tikz/note highlight}.
  \end{docKey}

  \begin{docKey}[fb/notes/NOTE][]{split}{={boolean}}{false}
    Split that specific note if set to true.
  \end{docKey}

  \begin{docKey}[fb/notes/NOTE][]{shade}{={boolean}}{false}
    Shade that specific note if set to true.
  \end{docKey}

  \begin{docKey}[fb/notes/NOTE][]{highlight}{={boolean}}{false}
    Highlight that specific note if set to true.
  \end{docKey}

\end{docCommand}

\begin{docCommand}{FBcopyNote}{\oarg{options}\marg{src}\marg{to}}
  Copies the note \marg{src} to \marg{to} and override parameters given in
  \oarg{options}. See \refCom{FBnewNote}.
\end{docCommand}

\begin{docCommand}{FBgetNote}{\marg{id}\marg{key}}
  Retrieves the \marg{key} value of the note \marg{id}.
\end{docCommand}


\subsection{Colors}

Colors are predefined (see annexes for the full list).


\begin{docCommand}{FBnewColor}{\oarg{options}\marg{name}\marg{spec}}
  Creates a new color \meta{name} using \meta{spec} specifications which are
  the same as the \pkg{xcolor} package. This command creates 2 colors, one
  for the background using \meta{spec} and one for the foreground which can
  be either black or white depending on the background color. The
  computation is based on the luminance\footnote{See
    \url{https://en.wikipedia.org/wiki/HSL_and_HSV\#Lightness}} value.

  You want to use either HTML or CMYK color definition. In case of HTML the
  color \meta{spec} is \brackets{RRGGBB} which is the hexadecimal color
  composition. If you choose CMYK the definition is \brackets{C, M, Y, K}
  for cyan, magenta, yellow and black with a 0 to 1 valid range.
  
  Some options can be passed to to \cs{FBnewColor} command:

  \begin{docKey}[fb/colors][]{limit}{=limit}{0.5}
    The luminance limit. If the background color's luminance if greater than
    \meta{limit}, the foreground color is set to black, white otherwise.

    You shouldn't change it unless you experiment some strange behaviors.
  \end{docKey}

  \begin{docKey}[fb/colors][]{model}{=model}{HTML}
    The color model to use. See the \pkg{xcolor} documentation for further
    information.
  \end{docKey}
  
  \begin{docKey}[fb/colors][]{derive}{=boolean}{false}
    If true, the new color is based on an existing color passed in the
    \meta{spec} argument. Otherwise, a new color is created.
  \end{docKey}
  
\end{docCommand}


\section{Tips}

Inserting a lot of diagrams in a document can take long time to build since
\pkg{tikz} is not very fast. If your diagrams do not change very often, you
can generate a standalone PDF and include it in your document.

A very simple example can be:

\begin{example}
% chords/c-major.tex
\documentclass[]{standalone}
\usepackage{guitar-fretboard}
\begin{document}
\begin{fretboard}[frets min=0, frets max=4, fret numbers,
    left handed, title = {\pC}]%
  \FBnote{C} \FBnote{E} \FBnote{G}
\end{fretboard}%
\end{document}
\end{example}

Now you can add the generated PDF using:

\begin{example}
  \includegraphics[width=0.7\textwidth,keepaspectratio]{chords/c-major.pdf}
\end{example}

\ifannexes
\section{Annexes}

\subsection{pitches}

All semitones are relative to \pC.

\begin{small}
  \centering
  \begin{longtable}{l|c|c|c|c|c|l}
    id &  degree & semitones & tex & tex small & reversed & command \\
    \hline
    \endhead
    \mypitchline{Cbbb}
    \mypitchline{Cbb}
    \mypitchline{Cb}
    \mypitchline{C}
    \mypitchline{CS}
    \mypitchline{CSS}
    \mypitchline{CSSS}
    \hline
    \mypitchline{Dbbb}
    \mypitchline{Dbb}
    \mypitchline{Db}
    \mypitchline{D}
    \mypitchline{DS}
    \mypitchline{DSS}
    \mypitchline{DSSS}
    \hline
    \mypitchline{Ebbb}
    \mypitchline{Ebb}
    \mypitchline{Eb}
    \mypitchline{E}
    \mypitchline{ES}
    \mypitchline{ESS}
    \mypitchline{ESSS}
    \hline
    \mypitchline{Fbbb}
    \mypitchline{Fbb}
    \mypitchline{Fb}
    \mypitchline{F}
    \mypitchline{FS}
    \mypitchline{FSS}
    \mypitchline{FSSS}
    \hline
    \mypitchline{Gbbb}
    \mypitchline{Gbb}
    \mypitchline{Gb}
    \mypitchline{G}
    \mypitchline{GS}
    \mypitchline{GSS}
    \mypitchline{GSSS}
    \hline
    \mypitchline{Abbb}
    \mypitchline{Abb}
    \mypitchline{Ab}
    \mypitchline{A}
    \mypitchline{AS}
    \mypitchline{ASS}
    \mypitchline{ASSS}
    \hline
    \mypitchline{Bbbb}
    \mypitchline{Bbb}
    \mypitchline{Bb}
    \mypitchline{B}
    \mypitchline{BS}
    \mypitchline{BSS}
    \mypitchline{BSSS}
    \hline
  \end{longtable}
\end{small}

\subsection{Intervals}

\begin{small}
  \centering
  \begin{longtable}{l|c|c|l|c|c|c}
    id & degree & semitones & name & tex & tex small & reversed\\
    \hline
    \endhead
    \myintervalline{d1}
    \myintervalline{b1}
    \myintervalline {1}
    \myintervalline{A1}
    \myintervalline{S1}
    \hline
    \myintervalline{bb2}
    \myintervalline {d2}
    \myintervalline {b2}
    \myintervalline {m2}
    \myintervalline  {2}
    \myintervalline {A2}
    \myintervalline {S2}
    \hline
    \myintervalline{bb3}
    \myintervalline {d3}
    \myintervalline {b3}
    \myintervalline {m3}
    \myintervalline  {3}
    \myintervalline {A3}
    \myintervalline {S3}
    \hline
    \myintervalline{d4}
    \myintervalline{b4}
    \myintervalline {4}
    \myintervalline{A4}
    \myintervalline{S4}
    \hline
    \myintervalline{d5}
    \myintervalline{b5}
    \myintervalline {5}
    \myintervalline{A5}
    \myintervalline{S5}
    \hline
    \myintervalline {d6}
    \myintervalline{bb6}
    \myintervalline {m6}
    \myintervalline {b6}
    \myintervalline  {6}
    \myintervalline {A6}
    \myintervalline {S6}
    \hline
    \myintervalline {d7}
    \myintervalline{bb7}
    \myintervalline {m7}
    \myintervalline {b7}
    \myintervalline  {7}
    \myintervalline {A7}
    \myintervalline {S7}
    \hline
    \myintervalline{d8}
    \myintervalline{b8}
    \myintervalline {8}
    \myintervalline{A8}
    \myintervalline{S8}
    \hline
    \myintervalline{d9}
    \myintervalline{bb9}
    \myintervalline{m9}
    \myintervalline{b9}
    \myintervalline {9}
    \myintervalline{A9}
    \myintervalline{S9}
    \hline
    \myintervalline{d11}
    \myintervalline{b11}
    \myintervalline {11}
    \myintervalline{A11}
    \myintervalline{S11}
    \hline
    \myintervalline {d13}
    \myintervalline{bb13}
    \myintervalline {m13}
    \myintervalline {b13}
    \myintervalline  {13}
    \myintervalline {A13}
    \myintervalline {S13}
    \hline
  \end{longtable}
\end{small}

\subsection{Chords}
%% \begin{small}
%%   \centering
  \begin{longtable}{l|l|l|l|l}
    id & name & tex & tex fs & interval  \\
    \hline
    \mychordline   {M}
    \mychordline   {m}
    \mychordline   {5}
    \mychordline   {b5}
    \mychordline   {S5}
    \mychordline   {sus2}
    \mychordline   {sus4}
    \mychordline   {aug}
    \mychordline   {dim}
    \mychordline   {7}
    \mychordline   {M7}
    \mychordline   {m7}
    \mychordline   {7sus4}
    \mychordline   {m7maj}
    \mychordline   {7majS5}
    \mychordline   {7S5}
    \mychordline   {7b5}
    \mychordline   {m7b5}
    \mychordline   {dim7}
    \mychordline   {6}
    \mychordline   {m6}
    \mychordline   {add9}
    \mychordline   {6add9}
    \mychordline   {maj9}
    \mychordline   {maj7S11}
    \mychordline   {maj13}
  \end{longtable}
%% \end{small}

\subsection{Tunings}
%% \begin{small}
%%   \centering
\begin{longtable}{l|l|l}
    id & Notes & semitones from \pC  \\
    \hline
    \mytuningline   {guitar/standard}
    \mytuningline   {guitar/dadgad}
    \mytuningline   {guitar/drop d}
    \mytuningline   {guitar/double drop d}
    \mytuningline   {guitar/open g}
    \mytuningline   {guitar/7 string}
    \hline
    \mytuningline   {bass/standard}
    \mytuningline   {bass/5 string}
    \mytuningline   {bass/5 string tenor}
    \mytuningline   {bass/6 string}
    \hline
    \mytuningline   {ukulele/standard}
\end{longtable}
  
  
\subsection{Colors}

\begin{small}
  \centering
  \begin{longtable}{l|c|c}
    id & RGB name & CMYK name  \\
    \hline
    \mycolorline{1bb}
    \mycolorline{1b}
    \mycolorline{1}
    \mycolorline{1s}
    \mycolorline{1ss}
    \hline
    \mycolorline{2bb}
    \mycolorline{2b}
    \mycolorline{2}
    \mycolorline{2s}
    \mycolorline{2ss}
    \hline
    \mycolorline{3bb}
    \mycolorline{3b}
    \mycolorline{3}
    \mycolorline{3s}
    \mycolorline{3ss}
    \hline
    \mycolorline{4bb}
    \mycolorline{4b}
    \mycolorline{4}
    \mycolorline{4s}
    \mycolorline{4ss}
    \hline
    \mycolorline{5bb}
    \mycolorline{5b}
    \mycolorline{5}
    \mycolorline{5s}
    \mycolorline{5ss}
    \hline
    \mycolorline{6bb}
    \mycolorline{6b}
    \mycolorline{6}
    \mycolorline{6s}
    \mycolorline{6ss}
    \hline
    \mycolorline{7bb}
    \mycolorline{7b}
    \mycolorline{7}
    \mycolorline{7s}
    \mycolorline{7ss}
    \hline
  \end{longtable}
\end{small}

\fi

\end{document}
